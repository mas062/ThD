\chapter{Scientific environment}

This dissertation is submitted as a partial fulfillment of the requirements for
the degree Doctor of Philosophy (PhD) at the University of Bergen. It is part
of the project Mathematical Modeling and Risk Assessment of CO2 storage,
MatMoRA, which is funded by the Norwegian Research Council, Statoil and Norske
Shell under grant no. 178013/I30 and lead by Professor Helge Dahle at the
Department of Mathematics, University of Bergen (UiB). 

The working environment have been SINTEF-ICT in Oslo, and CIPR in Bergen, and
SIMTECH in Stuttgart. The chief scientist at SINTEF-ICT, professor Knut-Andreas
Lie, has been the main adviser and Professor
Jan M. Nordbotten at the Department of Mathematics, UiB, along with the research
scientist at SINTEF, Halvor M. Nilsen, have been the co-advisers. With warm
supports from professor Rainer Helmig, head of department of hydromechanics and
modelling of hydrosystems at Stuttgart university, last parts of the work is
benefited from advices of professor Wolfgang Nowak, head of Stochastic modelling
of hydrosystems, Dr. Sergey Oladyshkin, postdoctoral fellow at SIMTECH,
and professor Holger Class at Stuttgart university. 